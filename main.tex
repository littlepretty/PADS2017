\documentclass{sig-alternate-05-2015}

% *** CITATION PACKAGES ***
\usepackage{cite}

% *** MATH PACKAGES ***
\usepackage{amsmath}
\usepackage{svg}
\usepackage{amsfonts}

% *** PDF, URL AND HYPERLINK PACKAGES ***
\usepackage{url}
\usepackage{hyperref}
\usepackage{epigraph}

% *** ALGORITHM ***
%\usepackage{algorithm}
%\usepackage{algpseudocode}
\usepackage[ruled, linesnumbered]{algorithm2e}
\renewcommand{\algorithmcfname}{ALGORITHM}
\SetAlFnt{\small}
\SetAlCapFnt{\small}
\SetAlCapNameFnt{\small}
\SetAlCapHSkip{0pt}
\IncMargin{-\parindent}

\usepackage{booktabs}
\usepackage{enumitem}
\usepackage{float}

% *** To balance reference page ***
\usepackage{flushend}

% *** Draw diagrams ***
\usepackage{tikz}
\usetikzlibrary{positioning}
\usetikzlibrary{arrows.meta}

% *** Fancy Table ***
\usepackage{multirow}

\newcommand{\paragraphb}[1]{\vspace{0.03in}\noindent{\bf #1} }
\newcommand{\paragraphe}[1]{\vspace{0.03in}\noindent{\em #1} }
\newcommand{\paragraphbe}[1]{\vspace{0.03in}\noindent{\bf \em #1} }

\newcommand{\kevin}[1]{{\color{red}{#1}}}
\newcommand{\kevinc}[1]{{\color{red}\bf\em{[kevin: #1]}}}

\newcommand{\name}{DSSnet\xspace}
\newcommand{\Name}{DSSnet\xspace}

\newtheorem{lemma}{Lemma}
\newtheorem{theorem}{Theorem}
\newdef{definition}{Definition}

% *** For commenting blocks of text ***
%\newcommand{\CUT}[1]{{}}
\begin{document}

%\begin{comment}
% DOI
%\doi{000-0-0000-0 00-0}

% ISBN
%\isbn{0000000/0000}

\CopyrightYear{2017} \setcopyright{acmcopyright}
\conferenceinfo{SIGSIM-PADS '17,}{May 24--26, 2017, Singapore.}
\isbn{xxx-x-xxxx-xxxx-x/xx/xx}\acmPrice{\$15.00}
\doi{http://dx.doi.org/xx.xxxx/xxxxxxxxxx.xxxxxx}
%Authors, replace the red X's with your assigned DOI string.
\clubpenalty=10000
\widowpenalty = 10000
%\end{comment}

%
% --- Author Metadata here ---
%\conferenceinfo{WOODSTOCK}{'97 El Paso, Texas USA}
%\CopyrightYear{2007} % Allows default copyright year (20XX) to be over-ridden - IF NEED BE.
%\crdata{0-12345-67-8/90/01}  % Allows default copyright data (0-89791-88-6/97/05) to be over-ridden - IF NEED BE.
% --- End of Author Metadata ---

\title{Simulation of a Software-Defined Network as One Big Switch}
%\subtitle{[BigSimSwitch]
%\titlenote{code for our project available on www.github.com}}

%
% You need the command \numberofauthors to handle the 'placement
% and alignment' of the authors beneath the title.
%
% For aesthetic reasons, we recommend 'three authors at a time'
% i.e. three 'name/affiliation blocks' be placed beneath the title.
%
% NOTE: You are NOT restricted in how many 'rows' of
% "name/affiliations" may appear. We just ask that you restrict
% the number of 'columns' to three.
%
% Because of the available 'opening page real-estate'
% we ask you to refrain from putting more than six authors
% (two rows with three columns) beneath the article title.
% More than six makes the first-page appear very cluttered indeed.
%
% Use the \alignauthor commands to handle the names
% and affiliations for an 'aesthetic maximum' of six authors.
% Add names, affiliations, addresses for
% the seventh etc. author(s) as the argument for the
% \additionalauthors command.
% These 'additional authors' will be output/set for you
% without further effort on your part as the last section in
% the body of your article BEFORE References or any Appendices.


%\begin{comment}
\numberofauthors{3} %  in this sample file, there are a *total*
% of EIGHT authors. SIX appear on the 'first-page' (for formatting
% reasons) and the remaining two appear in the \additionalauthors section.
%

\if 0 %removed temporarily for double-blinded review

\author{
% You can go ahead and credit any number of authors here,
% e.g. one 'row of three' or two rows (consisting of one row of three
% and a second row of one, two or three).
%
% The command \alignauthor (no curly braces needed) should
% precede each author name, affiliation/snail-mail address and
% e-mail address. Additionally, tag each line of
% affiliation/address with \affaddr, and tag the
% e-mail address with \email.
%
% 1st. author
\alignauthor
Jiaqi Yan\\
      \affaddr{Illinois Institute of Technology}\\
       \affaddr{10 West 31st Street}\\
       \affaddr{Chicago, Illinois, 60616 }\\
       \email{jyan31@hawk.iit.edu}
% 2nd. author
\alignauthor
Xin Liu\\
       \affaddr{Illinois Institute of Technology}\\
       \affaddr{10 West 31st Street}\\
       \affaddr{Chicago, Illinois, 60616 }\\
       \email{xliu125@hawk.iit.edu}
% 3rd. author
\alignauthor
Dong Jin\\
       \affaddr{Illinois Institute of Technology}\\
       \affaddr{10 West 31st Street}\\
       \affaddr{Chicago, Illinois, 60616 }\\
       \email{dong.jin@iit.edu}
%\and  % use '\and' if you need 'another row' of author names
%% 4th. author
%\alignauthor Lawrence P. Leipuner\\
%       \affaddr{Brookhaven Laboratories}\\
%       \affaddr{Brookhaven National Lab}\\
%       \affaddr{P.O. Box 5000}\\
%       \email{lleipuner@researchlabs.org}
%% 5th. author
%\alignauthor Sean Fogarty\\
%       \affaddr{NASA Ames Research Center}\\
%       \affaddr{Moffett Field}\\
%       \affaddr{California 94035}\\
%       \email{fogartys@amesres.org}
%% 6th. author
%\alignauthor Charles Palmer\\
%       \affaddr{Palmer Research Laboratories}\\
%       \affaddr{8600 Datapoint Drive}\\
%       \affaddr{San Antonio, Texas 78229}\\
%       \email{cpalmer@prl.com}
}
\fi

%\end{comment}

\maketitle
\begin{abstract}

Software-defined networking (SDN) technology promises centralized and rapid network provisioning,
holistic management, low operational cost, and improved network visibility.
Researchers have developed multiple SDN simulation and emulation platforms to expedite
the adoption of many emerging SDN-based applications to production systems.
However, the scalability of those platforms is often limited by the underlying physical
hardware resources, which inevitably affects the simulation fidelity in large-scale network settings.
In this paper, we present a model abstraction technique that effectively transforms
the network devices in an SDN-based network to one virtualized switch model.
While significantly enhancing the simulation scalability,
our abstracted model also preserves the end-to-end forwarding behavior of the original network.
To achieve this, we first classify packets with the same forwarding behavior into smaller
and disjoint groups by analyzing the OpenFlow rules installed on the SDN devices.
We then create a graph model for each group representing its forwarding behavior.
We traverse those graphs to aggregate the forwarding rules across the network,
and finally generate the effective rules for the big switch model.
Experimental results demonstrate that the network forwarding logic equivalence is well
preserved between the abstracted model and the original SDN network.
The model abstraction process is fast, e.g., 3.15 seconds to transform a medium-scale
tree network consisting of 53,260 rules.
The big-switch model is able to speed up the simulation by 4.3 times in average and
6.69 times in best case.


%Only one switch and 5766 forwarding rules are sufficient to model a tree-topology network with 40 switches and 53260 rules.

%Emulating the big-switch network has many benefits such as emulation resource reduction, larger-scale network support and component reuse.
%Our long term goal is to abstract the clustered portion of the SDN network so that
%emulating one OpenFlow switch is equivalent to a great number of interconnected
%OpenFlow switches in terms of both forwarding logics and network performance metrics.

\end{abstract}


%
% The code below should be generated by the tool at
% http://dl.acm.org/ccs.cfm
% Please copy and paste the code instead of the example below.
%
\begin{comment}
\begin{CCSXML}
<ccs2012>
 <concept>
  <concept_id>10010520.10010553.10010562</concept_id>
  <concept_desc>Computer systems organization~Embedded systems</concept_desc>
  <concept_significance>500</concept_significance>
 </concept>
 <concept>
  <concept_id>10010520.10010575.10010755</concept_id>
  <concept_desc>Computer systems organization~Redundancy</concept_desc>
  <concept_significance>300</concept_significance>
 </concept>
 <concept>
  <concept_id>10010520.10010553.10010554</concept_id>
  <concept_desc>Computer systems organization~Robotics</concept_desc>
  <concept_significance>100</concept_significance>
 </concept>
 <concept>
  <concept_id>10003033.10003083.10003095</concept_id>
  <concept_desc>Networks~Network reliability</concept_desc>
  <concept_significance>100</concept_significance>
 </concept>
</ccs2012>
\end{CCSXML}

\ccsdesc[500]{Computer systems organization~Embedded systems}
\ccsdesc[300]{Computer systems organization~Redundancy}
\ccsdesc{Computer systems organization~Robotics}
\ccsdesc[100]{Networks~Network reliability}

%
% End generated code
%
\end{comment}
%
%  Use this command to print the description
%
%\printccsdesc

% We no longer use \terms command
%\terms{Theory}

\keywords{Network Simulation; 	Model Abstraction; Software-Defined Networking;}

\section{Introduction}

Idea inspired by \cite{OneBigSwitchAbstraction}.

\section{Motivating Example}
\label{Sec:MotivatingExample}

\tikzset {
        roundnode/.style={circle, draw=blue!80, fill=blue!5, very thick, minimum size=7mm},
        squarenode/.style={rectangle, draw=red!80, fill=red!5, very thick, minimum size=7mm},
        ->-/.style={->, line width=1.2pt},
        ==/.style={line width=1.6pt}
}

\begin{figure}[t]
\centering
\begin{tikzpicture}
\node[roundnode] (sw0) {SW0};
\node[roundnode] (sw1) [below left=2cm of sw0] {SW1};
\node[roundnode] (sw2) [below=1.1cm of sw0] {SW2};
\node[roundnode] (sw3) [below right=2cm of sw0] {SW3};
\node[squarenode] (net1) [below=1cm of sw1] {10.1.0.10/16};
\node[squarenode] (net2) [below=1cm of sw2] {10.0.0.10/8};
\node[squarenode] (net3) [below=1cm of sw3] {11.1.0.10/16};

\draw[-, line width=1.2pt] (sw0) -- node [near start, above] {0} node[near end, below] {1} (sw1);
\draw[-, line width=1.2pt] (sw0) -- node [near start, right] {1} node[near end, left] {1} (sw2);
\draw[-, line width=1.2pt] (sw0) -- node [near start, above] {2} node[near end, below] {1} (sw3);
\draw[-, line width=1.2pt] (sw1) -- node [near start, left] {0} (net1);
\draw[-, line width=1.2pt] (sw2) -- node [near start, left] {0} (net2);
\draw[-, line width=1.2pt] (sw3) -- node [near start, right] {0} (net3);
\end{tikzpicture}
\caption{A Tree-Topology SDN Network}
\label{Fig:ExampleNetworkTopo}
\end{figure}

\begin{figure*}
\centering
\subfloat[Forwarding Graph for EC1 and EC3] {
\begin{tikzpicture}
\node[roundnode] (sw2) {SW2};
\node[roundnode] (sw0) [left=of sw2] {SW0};
\node[roundnode] (sw1) [above left=of sw0] {SW1};
\node[roundnode] (sw3) [below left=of sw0] {SW3};
\node[squarenode, align=center] (ec13) [right=of sw2] {EC1: 10.0.*.*\\EC3: 10.2.0.0-10.255.255.255};

\draw[->-] (sw0) -- node [near start, above] {1} node[near end, above] {1} (sw2);
\draw[->-] (sw1) -- node [near start, above] {1} node[near end, above] {0} (sw0);
\draw[->-] (sw3) -- node [near start, above] {1} node[near end, above] {2} (sw0);
\draw[->-] (sw2) -- node [near start, above] {0} (ec13);
\end{tikzpicture}
\label{Fig:ExampleForwardingEC13}
}

\subfloat[Forwarding Graph for EC2] {
\begin{tikzpicture}
\node[roundnode] (sw1) {SW1};
\node[roundnode] (sw0) [left=of sw2] {SW0};
\node[roundnode] (sw2) [above left=of sw0] {SW2};
\node[roundnode] (sw3) [below left=of sw0] {SW3};
\node[squarenode] (ec2) [right=of sw1] {EC2: 10.1.*.*};

\draw[->-] (sw0) -- node [near start, above] {0} node[near end, above] {1} (sw1);
\draw[->-] (sw2) -- node [near start, above] {1} node[near end, above] {1} (sw0);
\draw[->-] (sw3) -- node [near start, above] {1} node[near end, above] {2} (sw0);
\draw[->-] (sw1) -- node [near start, above] {0} (ec2);
\end{tikzpicture}
\label{Fig:ExampleForwardingEC2}
}
\subfloat[Forwarding Graph for EC4] {
\begin{tikzpicture}
\node[roundnode] (sw3) {SW3};
\node[roundnode] (sw0) [left=of sw3] {SW0};
\node[roundnode] (sw1) [above left=of sw0] {SW1};
\node[roundnode] (sw2) [below left=of sw0] {SW2};
\node[squarenode] (ec4) [right=of sw3] {EC4: 11.1.*.*};

\draw[->-] (sw0) -- node [near start, above] {2} node[near end, above] {2} (sw3);
\draw[->-] (sw1) -- node [near start, above] {1} node[near end, above] {0} (sw0);
\draw[->-] (sw2) -- node [near start, above] {1} node[near end, above] {1} (sw0);
\draw[->-] (sw3) -- node [near start, above] {0} (ec4);
\end{tikzpicture}
\label{Fig:ExampleForwardingEC4}
}
\caption{Forward Graph for Each EC}
\label{Fig:ExampleForwardingGraphs}
\end{figure*}

\begin{table*}[t]
\caption{Forwarding rules on each OpenFlow switch in the 3-ary tree topology network}
\begin{center}
\begin{tabular}{c|clc}
\hline
Switch & Priority & Match Field & Action\\
\hline
\hline
\multirow{3}{2em}{SW0} & 10 & NW\_DST=10.1.*.* & FWD: OUT\_PORT=0 \\
                       & 1  & NW\_DST=10.*.*.* & FWD: OUT\_PORT=1 \\
                       & 1  & NW\_DST=11.1.*.* & FWD: OUT\_PORT=2 \\
\hline
\multirow{3}{2em}{SW1} & 10 & IN\_PORT=1, NW\_DST=10.1.*.* & FWD: OUT\_PORT=0 \\
                       & 1  & IN\_PORT=0, NW\_DST=10.*.*.* & FWD: OUT\_PORT=1 \\
                       & 1  & IN\_PORT=0, NW\_DST=11.1.*.* & FWD: OUT\_PORT=1 \\
\hline
\multirow{3}{2em}{SW2} & 10 & IN\_PORT=0, NW\_DST=10.1.*.* & FWD: OUT\_PORT=1 \\
                       & 1  & IN\_PORT=1, NW\_DST=10.*.*.* & FWD: OUT\_PORT=0 \\
                       & 1  & IN\_PORT=0, NW\_DST=11.1.*.* & FWD: OUT\_PORT=1 \\
\hline
\multirow{3}{2em}{SW3} & 10 & IN\_PORT=1, NW\_DST=11.1.*.* & FWD: OUT\_PORT=0 \\
                       & 1  & IN\_PORT=0, NW\_DST=10.*.*.* & FWD: OUT\_PORT=1 \\
                       & 1  & IN\_PORT=0, NW\_DST=10.1.*.* & FWD: OUT\_PORT=1 \\
\hline
\end{tabular}
\end{center}
\label{Tab:OriginalFlowTable}
\end{table*}

\begin{table*}[t]
\caption{Forwarding Rules on the ``Big OpenFlow Switch"}
\begin{center}
\begin{tabular}{c|clc}
\hline
Switch & Priority & Match Field & Action\\
\hline
\hline
\multirow{4}{2em}{SW}  & 10 & NW\_DST=10.0.*.* & FWD OUT\_PORT=1 \\
                       & 10 & NW\_DST=10.1.*.* & FWD OUT\_PORT=0 \\
                       & 10 & NW\_DST=11.1.*.* & FWD OUT\_PORT=2 \\
                       & 10 & NW\_DST=10.2.0.0-10.255.255.255 & FWD OUT\_PORT=1 \\
\hline
\end{tabular}
\end{center}
\label{Tab:CompressedFlowTable}
\end{table*}

In this section, we describe our idea in a concrete network example and
walk through our method of how to compressing a SDN network to a ``big switch" module.
Let's consider the tree-topology network connected by 4 OpenFlow switches,
as shown in Figure~\ref{Fig:ExampleNetworkTopo}.
Centralized SDN controller(not shown in the figure) installs the rules shown in
Table~\ref{Tab:OriginalFlowTable} to each switch so that connection is established
for three subnets.
We assume the network has been in a stationary state, e.g. rules are already available
on each OpenFlow switch and there is no dynamic events occur such as link down
or rule modification.
SW0 then works properly as an \textbf{aggregate switch} to route traffic
between different subnets.
SW1, SW2 and SW3 play as \textbf{edge switches}, switches at the leave of the topology,
for three subnets respectively.
In this very special case, there is only one host connected to each edge switch.

Our approach abstracts the network composed of both aggregate switch and edge switches
to one ``big switch" that has logically equivalent functions.
The first step in the abstraction process is to extract \textit{equivalence classes}
through the OpenFlow rules from any devices in the network.
As formally defined later in Section~\ref{Sec:Design} and in \cite{Veriflow}, equivalence class (EC) is
the set of packets that experience identical forwarding action at \textbf{any} network device.
This concept is proposed to confine network verification activities to the minimal
effected set of packets\cite{Veriflow}.
In our approach, we utilize EC to merge all the rules on a set of switches.
For example, the flow rules shown in Table~\ref{Tab:OriginalFlowTable} can be sliced into
4 \textbf{disjoint} ECs based on the NW\_DST match field:
\begin{itemize}
\item Packets in EC1 is destined to network address 10.0.*.*.
\item Packets in EC2 is going to hosts with address 10.1.*.*.
\item Packets going to the address range from 10.2.0.0 to 10.255.255.255 belongs to EC3
\item Packets in EC4 goes to subnet 11.1.*.*. 
\end{itemize}
One thing to notice is that match field IN\_PORT cannot be used in identifying ECs,
since it is not packet-dependent but topology-dependent filed.

After identifying all the equivalence classes from the rule set,
we generate \textit{forwarding graph} which models how packets within an EC will be
forwarded through the network\cite{Veriflow}.
The node in a forwarding graph represents EC at a network device;
the directed edge represents how the network device forward the EC.
The sink nodes (shown as red rectangle node) is not part of the forwarding graph
but are drawn for recognizing which EC this forwarding graph belongs to.
If we skip the process of minimizing number of ECs from the rules,
each equivalence class will have exactly one forwarding graph.
Forwarding graph for all four ECs are shown separately in
Figure~\ref{Fig:ExampleForwardingGraphs}.
Notice that \{EC1, EC2, EC3, EC4\} is not the minimal set of ECs in the network.
In fact, EC1 and EC3 can be merged together because packets in both classes are forwarded
identically at any network device.
This is shown in Figure~\ref{Fig:ExampleForwardingEC13} as EC1 and EC3 share the same
forwarding graph.

We finish the abstraction by generating a new set of forwarding rules that are to be
installed on the ``big switch".
This process can be efficiently by considering only these ECs that
traverses edge switches in the abstracted network.
Table~\ref{Tab:CompressedFlowTable} shows the resulting rules that will be installed in
SW in Figure~\ref{Fig:ExampleBigSwitch}.

\begin{figure}[t]
\centering
\begin{tikzpicture}
\node[roundnode, minimum size=15mm, align=center] (sw) {Big\\Switch};
\node[squarenode, align=center] (ec13) [below=1.3cm of sw] {10.0.0.10/8};
\node[squarenode] (ec2) [below left=2cm of sw] {10.1.0.10/16};
\node[squarenode] (ec4) [below right=2cm of sw] {11.1.0.10/16};

\draw[-, line width=1.2pt] (sw) -- node [near start, above] {0} (ec2);
\draw[-, line width=1.2pt] (sw) -- node [near start, right] {1} (ec13);
\draw[-, line width=1.2pt] (sw) -- node [near start, above] {2} (ec4);
\end{tikzpicture}
\caption{Compressed SDN Network for Scalable Simulation}
\label{Fig:ExampleBigSwitch}
\end{figure}

The resulting one-switch network functions identical to the tree-topology network,
in the eyes of the three hosts(possibly emulated or simulated).
In contrast, the number of switches we need to emulate is reduced from four to one;
more importantly, the number of rules in the network is reduced from twelve to four.
If we confine the rules to these that (1) only match NW\_DST field;
(2) only have forwarding actions,
the total number of rules will be proportional to the number of ECs in the
original SDN network, comparing to $O(S\times P)$ where $S$ is the number of switches
and $P$ is the number of address prefixes.



%\section{Transformation of an SDN network to an OpenFlow switch}
\section{SDN Model Abstraction}
\label{Sec:Design}

\begin{comment}
The overview of our model abstraction approach for SDN networks is shown in Figure~\ref{Fig:BigSimOverview}. Through a three-step systematic process, we are able to take 
a snapshot of an SDN network $SN$ as the input, transform the network devices in an SDN network to a single OpenFlow switch, $BS$ that preserves the forwarding logic of the original network. In this section, we elaborate the algorithmic design in each step of the model abstraction.
%We have sketched the process of how to replace a OpenFlow-switch-connected network with a single OpenFlow switch in section~\ref{Sec:MotivationalExample}.
\end{comment}

Our objective is to effectively transform a static SDN data plane configuration (i.e., a snapshot of the network state) to one ``big switch" model, which preserves the same end-to-end forwarding behavior. %In other words, if a packet originated from a host is forwarded to host in the snapshot, then the packet will be sent to the same destination by the ``big switch". 
To achieve this objective, we need to identify how every packet is processed in the snapshot, and how to correctly configure the big switch model to reflect the identical forwarding logic. In this paper, we develop a three-step model abstraction method, which is summarized as follows.

\begin{itemize}
\item \textbf{Identifying Equivalence Classes.} We partition all possible packets in the network into mutually exclusive sets (i.e., equivalence class, as formally defined in Section \ref{ec}), and the packets belongs to the same set are processed in the same way. Those sets are identified according to the matching field of \textbf{all} the OpenFlow rules on \textbf{all} the SDN switches in the original network.

\item \textbf{Creating Forwarding Graphs.} We model the forwarding behavior of each packet set using the topology information as well as the local information stored on SDN switches (e.g., port mapping, rule priorities, etc), and generate a graph-based model to represent the forwarding behavior.  
\item \textbf{Generating OpenFlow Rules of the Big Switch.} We generate the OpenFlow rules for the big switch in order to preserve the end-to-end forwarding logic. This step includes (1) constructing the port-to-host mapping, (2) generating the rules by matching the packet header of each set,  and (3) forwarding the packet to the correct output port, which is determined by traversing the forwarding graph acquired in step 2.
\end{itemize}

Our three-step approach has two assumptions. First, the controller can dynamically change the configuration of each network device, but we assume that the frequency of issuing such control messages is far less than the rate of the incoming packets. Between two configuration updates, the data plane remains unchanged. Therefore, we can exclude the SDN controller from the abstracted network model. Second, we do not consider packet header modification actions on the network device. We describe each step in details in the remainder of the section.

%We assume that both the original network and abstracted network use OpenFlow switches, and the switches compare the incoming packet to the installed rule with highest priority and then apply the action of that rule. The simplicity of OpenFlow switches enables us to analyze the behavior of the original network and synthesis the configuration of the ``big switch". 
   
\subsection{Identifying Equivalent Classes}
\label{ec}

\begin{comment}
By aggregating and slicing forwarding rules in an SDN network according to ECs, one can obtain a compact representation of the network state in terms of forwarding behavior. 
\end{comment}

We first give the definition of equivalence class (EC), and then present the data structure and algorithms to partition the packets into ECs.

\begin{definition}
An equivalence class is a set of packets that experience the identical forwarding action at \textbf{any} network device in the network. 
\label{Def:EC}
\end{definition}

\begin{comment}
%\kevin{add to one or two sentences about how to find ECs and how to aggregate ECs before the structure}
By its definition, ECs can be found by aggregate forwarding rules on different switches
and group them by the match field.
We aggregate forwarding rules using a trie structure as inspired by VeriFlow\cite{Veriflow}. 
A trie node has three child nodes, i.e., zero, one or wildcard, that represent three possible values for performing a bit-to-bit rule matching.
The entire tree is composited by several sub-tries, each representing one packet header field. For example, consider the tree-topology network in Section \ref{Sec:MotivatingExample}, the trie only contains one sub-trie representing the $NW\_DST$ header field (i.e., the network destination address). Note that an EC can be defined by multiple fields. For example, we can also match the source address of the packet (e.g., $NW\_SRC$) as well as the service type of the traffic ($TCP\_SRC$). The resulting trie now has three sub-tries.
%Here we give solutions to two problems that VeriFlow neglected.
\end{comment}

Each packet is uniquely identified by its header field values, which are matched against the forwarding rules in the OpenFlow switches to determine the appropriate action. Since the matching fields of the OpenFlow rules typically contain the wildcard suffix (e.g., longest prefix match of IP source/destination addresses), a group of packets with consecutive header values are often processed by the same rule. We use a trie structure, originally proposed by VeriFlow \cite{Veriflow}, to maintain the matching fields of all the OpenFlow rules in the network. The trie is composed of several sub-tries, and each sub-trie stands for a matching field (e.g., source/destination MAC address/IP address/port, etc.). Each node in sub-trie presents one bit in the corresponding matching field, and each node has three edges to the next node (i.e., next bit in the  matching field). The edges represent three possible bit-to-bit rule matching conditions: zero, one, or wildcard (i.e., don't care). The rule metadata are stored in the corresponding leaf node, including the rule's location (switch\#), action (forwarding to an out port or dropping the packet), priority, etc. 

Having all the OpenFlow rules inserted in the aforementioned trie structure, we perform the following three steps to identify the equivalence classes in the network. (1) We traverse the trie to obtain the consecutive header values for each rule. (2) After having a collection of header value intervals, which are denoted by the starting and end values, we develop an algorithm to split the existing intervals into smaller and non-overlapping intervals. Each non-overlapping interval identifies the packets belonging to an equivalence class. (3) We merge certain equivalence classes in order to reduce the time and space complexity for the forwarding graph generation (Section \ref{Sec:Generating Forwarding Graphs}) and the big switch rule generation (Section \ref{sec:thirdstep}). The details of second and third steps for EC identification are presented as follows.

\if 0
\begin{figure}[t]
\centering
\includegraphics[scale=.35]{figures/trie.pdf}
\caption{A trie data structure to maintain header values.}
\label{Fig:Trie}
\end{figure}
\fi 

\subsubsection{Splitting Overlapping Intervals}
% Xin: I changed all the "ranges" into "intervals" or "header intervals", since the term "range" has a specific meaning in maths.
By traversing from the root node to all leaf nodes, we obtain a set of packets header intervals that matched by all the rules. Each interval is represented by a pair of starting and ending values as $A, B$ and $C$ in Figure~\ref{Fig:DisjointECsAsInterval}. Identifying a set of equivalence classes requires us to split this list of intervals $I$ to a list of non-overlapping ones, each of which can be seen as an EC. An example is shown in the upper part of Figure~\ref{Fig:DisjointECsAsInterval}. 
We develop Algorithm~\ref{Alg:GenDisjointECs} to generate a set of disjoint intervals and show that the generation can be accomplished in $O(N \times M\log M)$ time,
where $M=|I|$ is the number of intervals and $N$ is the number of header bits.
% as follows\cite{SplitDisjointInterval}.
%Not algorithmic described in\cite{Veriflow},

First, we place $I$ into an array $A$ of $2M$ values,
each flagged as either a start point, or an end point, of an interval
\footnote{Equal values with same flag are reduced to an single element.}.
Before iterating $A$, we sort it, breaking the tie by putting start point before end point.
Then we maintain the difference $d$ between the number of seen start points and
the number of seen end points. While visit each point in sorted order:
\begin{itemize}
\item If the current element $x \in S$ and $d > 0$,
        we end the previous interval with the ending value $x - 1$;
        Start a new interval with a starting value $x$
        (line~\ref{Alg:LineEndStart1}-\ref{Alg:LineEndStart2}).
\item If the current element $x \in E$, we end the previous interval with the ending value $x$.
        (line~\ref{Alg:LineEndEnd}).
\item In either case, we update the potential new interval's start value \textit{prev}
        (line~\ref{Alg:LineNewPrev1} and \ref{Alg:LineNewPrev2}).
\end{itemize}

Inserting/deleting/modifying forwarding rules in the network will change the EC set.
By maintaining the rules in a trie, we can efficiently update ECs in an incremental way. An insertion of a new rule requires us to do a depth-first-style trie traverse. This process automatically narrows down the set of affected rules by ignoring those non-overlapping branches with the new rule.
The output is the set of the affected intervals, and we need to run Algorithm~\ref{Alg:GenDisjointECs} to update ECs only in the affected ones.


\begin{figure}[t]
\centering
\includegraphics[scale=.52]{figures/DisjointECs.pdf}
\caption{A class of packets are abstracted as an interval of packet header values.
        Five equivalence classes, shown at the top, can be obtained via splitting three intervals A, B and C.
        Finding $\Delta[EC_x]$, the rules that intersect with EC $x$, is instrumental for
        merge equivalent ECs, which are shown at the bottom.}
\label{Fig:DisjointECsAsInterval}
\end{figure}

\begin{algorithm}[t]
\DontPrintSemicolon
\KwData{$I = $ a set of packet header intervals from the leaves of the trie}
\KwResult{$EC = $ a set of equivalence classes as disjoint intervals}
$cnt \gets 0$\;
$S = $ \{starting points of $\forall i \in I\}$, $E = $ \{end points of $\forall i \in I\}$\;
$A \gets Sort(S \bigcup E)$ in non-decreasing order\;
$EC \gets \emptyset$\;
\ForEach {$x \in A$} {
        \uIf {$x \in S$} {
                \If {$cnt \neq 0$} {\label{Alg:LineEndStart1} 
                        $EC \gets EC \text{ }\bigcup \text{} [prev, x-1]$\;
                }\label{Alg:LineEndStart2} 
                $prev \gets x$\;\label{Alg:LineNewPrev1}
                $cnt \gets cnt + 1$\;
        }
        \Else ($x \in E$) {
                $EC \gets EC \text{ } \bigcup \text{ } [prev, x]$\;\label{Alg:LineEndEnd}
                $prev \gets x + 1$\;\label{Alg:LineNewPrev2}
                $cnt \gets cnt - 1$\;
        }
}
\caption{Splitting Overlapping Intervals}
\label{Alg:GenDisjointECs}
\end{algorithm}

\subsubsection{Combining Equivalent Classes}
The collection of ECs (i.e., disjoint intervals) obtained from Algorithm~\ref{Alg:GenDisjointECs} are able to be combined. 
We can union some ECs if they represent the identical packet forwarding behavior (see the definition of ECs). 
For example, $EC_2$ and $EC_4$ in Figure~\ref{Fig:DisjointECsAsInterval} can be combined as one EC, since the packets in both ECs experience the same set of forwarding rules throughout the network. %To generate the correct set of rules on the new ``big switch", we only need to model the forwarding behavior of either one of the ECs. In other words, for any two ECs $\alpha$ and $\beta$, we have:

\begin{lemma}
If a packets in EC $\alpha$ and another packet in EC $\beta$ experience the same forwarding actions on all switches, then $\alpha \cup \beta$ is also an EC.
\label{Lemma:MergeFG}
\end{lemma}
Combining two EC into a big one reduce the running time in the next two phases: generating forwarding graph and populating final OpenFlow rules. Also, the number of the resulting forwarding rules in the ``big switch" can be reduced. Here we provide an approach to identify whether two ECs can be unioned, which is built on the following lemma.

\begin{lemma}
For any two ECs $\alpha$ and $\beta$, they can be unioned into one EC if the packets have their header values covered by the same set of rules in the network.
\label{Lemma:MergeEC}
\end{lemma}
%According to Lemma~\ref{Lemma:MergeEC}, 
For example, both $EC_2$ and $EC_4$ are covered by interval $B$ and $C$ in Figure~\ref{Fig:DisjointECsAsInterval}, and therefore, we can treat them as one EC. The explanation is illustrated below.

First, we define a function $\Delta(x)$ that maps an EC $x$ to a set of forwarding rules that have their match field cover the header values of all the packets in $x$. So, if we have $\Delta(\alpha) = \Delta(\beta)$, then, 
for any network device $d$, let $\delta \in \Delta(\alpha)$ be the rule on $d$ with the highest priority.
If no such $\delta$ exists, packets from both $\alpha$ and $\beta$ are dropped on $d$.
Otherwise, packets in both $\alpha$ and $\beta$ match rule $\delta$, and
device $d$ will forward both of them according to the action specified in $\delta$.
Note that in another device $d'$, the highest priority rule that covers both $\alpha$
and $\beta$ may be different, i.e., $\delta' \neq \delta$.
However, as long as $\delta$ is unique at given $d$, the forwarding behavior at $d$ for both $\alpha$ and $\beta$ are always identical.

Given an EC, we can efficiently calculate $\Delta(\alpha)$ using two data structures: an array of pointers and a central interval tree.
Each of them is responsible for one of the two cases according to \cite{FindIntersectionWiki}.
\begin{itemize}
\item Case 1: A rule $\delta$ overlaps with an EC $\alpha$ with its starting and/or end point in $\alpha$.
        We can reuse the sorted array $A$ in Algorithm~\ref{Alg:GenDisjointECs}.
        We augment each value, either a start point or a end point of a interval, in $A$ with a pointer to the rule that the value belongs to.
        By doing a binary search, we can find the minimum and maximum values in $A$,
        which bound the interval of $\alpha$.
        All intervals that overlap with $\alpha$ must be between the minima and maxima,
        meaning we can ignore two kinds of rules: these whose end point is
        smaller than minima and these whose start point is larger than maxima.
        We then perform a linear search in the size-reduced set of rules,
        check one by one if the interval overlaps with $\alpha$.
        The total time complexity for both linear search and binary search are $O(\log M + K)$,
        where $K=|\Delta|$ is the number of reported intervals.
\item Case 2: Rule $\delta$ covers $\alpha$ entirely. We can build
        a central interval tree\cite{ComputationalGeometryBook} using all the available intervals.
        We pick a random value $x \in \alpha$ and query the central interval tree for
        all the ranges that intersect with $x$, which can be done in $O(\log M + K)$ time,
        at the cost of $O(M log M)$ time for building the central interval tree.
        Since central interval tree support efficient incremental operations (insertion and deletion),
        dynamic changes of rule set are also supported.
\end{itemize}

Using interval tree and ordered list, for each EC $\alpha$,
we find calculate $\Delta(\alpha)$ by mapping each rule $\delta \in \Delta(\alpha)$ to a unique binary ID $c_\delta$ of length $\log_2 M$,
we can encode $\Delta(\alpha)$ to a string of $c_\delta$s, putting small ID at front.
This string of unique IDs, named $C_\alpha$, is of at most length $M\log_2 M$.
We then use a hash table $H$ to group the mergeable ECs by hashing each EC $x$ to $C_x$.
The minimal size of ECs is the number of unique keys in $H$.
Note that in the subsequent algorithmic designs, iterating through all ECs refers to iterating through the first ECs in each set $H[key]$.%, $\forall key \in H$.

\subsection{Generating Forwarding Graphs}
\label{Sec:Generating Forwarding Graphs}

In the second step, we compute a forwarding graph for each EC, and then effectively reduce the size of the forwarding graph to improve efficiency for the third step. 

First, we define a function $FG(\alpha)$ that maps an EC $\alpha$ to a corresponding forwarding graph. A forwarding graph is a directed graph that represents how packets belonging to the same EC are processed by the network. A node $u$ in the forwarding graph is a networking device, and an edge $(u, v)$ in the graph means that device $u$ forwards the packets to device $v$ in the network. 
A forwarding graph not only concatenates the forwarding behavior for each EC, but also visualizes the data flow of the EC in the network. 
%For a fixed EC $x$, we connect the network devices that have rules for $x$ with directed edges that point to the next hop, which is determined by the action field of the rule.
Since our objective is to abstract the network forwarding logic into a big switch, our end-to-end modeling focuses on the sources and sinks of the graph. 
Figure~\ref{Fig:ForwardingGraphECX} depicts the generalized forwarding graph $FG(x)$ for EC $x$. 
 
%Here we discuss two specific issues in the forwarding graph generation: where to start the traversal according to our needs and what we can achieve at the end of the traversal.

\begin{figure*}[t]
\centering
\includegraphics[scale=.75]{figures/ForwardingGraph.pdf}
\caption{Modeling a Forwarding Graph of an Equivalent Class}
\label{Fig:ForwardingGraphECX}
\end{figure*}

\subsubsection{Network Traversal for Forwarding Graph Generation}

We develop a forwarding graph generation algorithm as shown in Algorithm~\ref{Alg:GenForwardingGraph}.
The notations are defined as follows.
$FG(x)$ denotes the forwarding graph for a particular equivalent class x.
A \textit{edge switch} is defined as a switch that has at least one link to a node outside $SN$.
A \textit{non-edge switch} is defined as a switch whose connected nodes are all inside $SN$.
The forwarding behavior of the non-edge switches are remove in the big switch abstraction.
$src$ denotes the source node of the forwarding path for an EC,
and $snk$ denotes the sink node of the forwarding path for an EC.
Note that all $src$ in $FG(x)$ are edge switches,
and $snk$ in $FG(x)$ can be either edge switches or non-edge switches.
$curr$ denotes the current node in the network that we are traversing to generate the forwarding graph.
We add a super-source node, $SRC^x$, and a super-sink node, $SNK^x$, as the boundaries of $FG(x)$.

Algorithm~\ref{Alg:GenForwardingGraph} is designed to generate $FG(x)$.
We start the process from each $src$ that connects to $SRC^x$,
and then traverse EC $x$'s forwarding graph using depth-first-based-search and
follow the action specified in the highest-priority forwarding rule for EC $x$ at each node along the way. 

%(2) the inner graph may contain both edge and internal switches.
%Source node $src$ and its out-going edge represent an edge switch $sw$ that forwards EC $x$ \textbf{coming from} port $p$; it can be described by $(sw, p)$ pair. Sink node $snk$ represent the end of the forwarding $sw$, which is also denoted by $(sw, p)$, where $p$ is either the port number specified by the action field or \texttt{NULL} if there is no rule for $x$ on $sw$.
%We denote the set of source nodes as $SRC(x)$, the set of sink nodes as $SNK(x)$.

We distinguish two kinds of port on an edge switch:
\begin{itemize}
\item \textit{end port} that connects to a node that is either the forwarding end point or outside the network one wants to abstract
\item \textit{inner port} that connects to a node inside the network that one wants to abstract.
\end{itemize}

We add an edge from $SRC^x$ to a $src$, if the source node has a forwarding rule $r$ that matches EC $x$, or the $IN\_PORT$ field of rule $r$ on the source node is an end port. 
Otherwise, we do not need to initiate a traverse (see line~\ref{Alg:LineStartDFS1} to~\ref{Alg:LineStartDFS2} in Algorithm~\ref{Alg:GenForwardingGraph}).
Correspondingly, we add an edge from a $snk$ to the super sink $SNK^x$ if both conditions are satisfied:
\begin{enumerate}
\item the sink node is an edge switch in the network;
\item the $OUT\_PORT$ field determined by the rule's action on the sink node is an end port.
\end{enumerate}

\begin{algorithm}[h]
\DontPrintSemicolon
\KwIn{$nodes = $ Switches containing rules for EC $x$ \newline
        $topo = $ Network topology}
\KwResult{Forwarding graph $FG(x)$ for EC $x$}
\SetKwProg{Fn}{Function}{}{\KwRet}
\SetKwFunction{Traverse}{traverse}
\SetKwFunction{GenRule}{generate\_rules}
\Fn{\Traverse{$curr$, $src$, $snk$}} {
        \uIf {$curr$ \upshape is \textbf{NOT} visited} {
                $r \gets$ \textbf{highest-priority} rule on $curr$ that processes EC $x$\;
                \If {r \upshape is NULL or $r.action$ is DROP} {\label{Alg:LineDropPath1}
                        $snk \gets$ ($curr$, NULL)\;
                        \GenRule{$x, src, snk$}\;
                        \KwRet\;
                }\label{Alg:LineDropPath2}
                $next \gets topo[curr][r.action.outport]$\;
                \If {next $\not\in$ nodes} {\label{Alg:LineForwardPath1}
                        $snk \gets$ ($curr$, $r.action.outport$)\;
                        \GenRule{$x, curr, src, snk$}\;
                        \KwRet\;
                }\label{Alg:LineForwardPath2}
                mark $curr$ as visited\;
                \Traverse{$next, src, snk$}\;
        }
        \Else {
                report forwarding loop\;\label{Alg:LineLoopPath}
        }
}\;
\ForEach{$n \in$ \upshape neighbors of $SRC^x$} {\label{Alg:LineStartDFS1}
        \If {\upshape $n$ is \textbf{NOT} visited} {
                $inport \gets$ input port number from $SRC^x$ to $n$\;
                \Traverse{$n$, $src=$\upshape($n$, $inport$), $snk=$NULL}\;
        }
}\label{Alg:LineStartDFS2}
\caption{Generating a Forwarding Graph for EC $x$\label{Alg:GenForwardingGraph}}
\end{algorithm}


\subsubsection{Network Traversal Outcomes}
After running Algorithm~\ref{Alg:GenForwardingGraph}, we can discover three kinds of ``path" in $FG(x)$ that are useful for the forwarding rule generation process on the big switch, i.e.,the third step of our network abstraction process (see Section \ref{sec:thirdstep}). 

\begin{itemize}
\item \textbf{Forwarding path} (line~\ref{Alg:LineForwardPath1}-\ref{Alg:LineForwardPath2}).
        The path from the super source node to the super sink node.
        This is a normal forwarding path for packets $\in$ EC $x$.
        
\item \textbf{Dropping packets in the network} (line~\ref{Alg:LineDropPath1}-\ref{Alg:LineDropPath2}).
        The path ends at a device inside the network, and fails to reach the super sink node. This indicates that the packets in EC $x$ are dropped inside the network.
        
\item \textbf{Forwarding loop}(line~\ref{Alg:LineLoopPath}).
        There is a directed cycle in the graph. One can simulate a forwarding loop in the network by (1) adding a rule in the big switch to drop the looping packets; or (2) dynamically monitoring the volume of the looping packets and adjusting the delay of looping packets and other packets sharing the communication path. We choose the first method since the model abstraction in the paper is focus on forwarding logic equivalence, and will leave the second method as future work when investigating forwarding performance equivalence. 
        %This behavior can be emulated in the semantic of \textbf{performance equivalence}, but not by \textbf{logical equivalence} studied in this paper.        
        %(1) recording the volume of the looping packets;
        %(2) increasing the delay of other packets on the basis of the amount of looping packets
\end{itemize}

\subsection{Populating the Flow Table on the Big Switch}\label{sec:thirdstep}

\begin{algorithm}[htbp]
\DontPrintSemicolon
\KwData{$PortMap$, which maps a $port$ on $sw$ to a $port$ on the big switch\newline
        $global\_port$, for port number assignment, and is initialized to 0}
\KwResult{A new rule $r$ to install on the big switch}
\SetKwProg{Fn}{Function}{}{\KwRet}
\SetKwFunction{GenRule}{generate\_rules}
\Fn{\GenRule{$x, src, dst$}} {
        $r.match \gets x$\;\label{Alg:LineMatch}
        \If {src.port $\not\in$ PortMap[src.sw]} {
                $PortMap[src.sw][src.port] \gets global\_port++$\;
        }
        $r.inport = PortMap[src.sw][src.port]$\;\label{Alg:LineInport}
        \uIf {dst.port \upshape is NULL} {
                $r.action \gets $ drop\_action\;\label{Alg:LineGenDropRule}
        }
        \Else {
                \If {dst.port $\not\in$ PortMap[dst.sw]} {\label{Alg:LineGenForwardRule1}
                        $PortMap[dst.sw][dst.port] \gets global\_port++$\;
                }
                $r.action \gets $ forward\_action\;
                $r.action.outport \gets PortMap[dst.sw][dst.port]$\;\label{Alg:LineGenForwardRule2}
                
        }
}
\caption{Generating Forwarding Rules for EC $x$ on the Big Switch\label{Alg:GenAllRules}}
\end{algorithm}

We develop an algorithm to generate OpenFlow rules on the big switch to abstract the forwarding behavior (see Algorithm~\ref{Alg:GenAllRules}).
%by calling \texttt{generate\_rules}, as described in Algorithm~\ref{Alg:GenAllRules}.
We maintain a hash table $PortMap$ to map the end ports of the edge switches to the ports of the big switch.
This table is configured using $global\_port$ variable during the rule generation procedure. 
Algorithm~\ref{Alg:GenAllRules} generates the mandatory fields in an OpenFlow rule:
\begin{itemize}
\item The $MATCH$ field is given by the EC $x$ itself, i.e., the range of matching packets header (line \ref{Alg:LineMatch});
\item The $IN\_PORT$ field is the mapped port number of $src.port$ (line~\ref{Alg:LineInport});
\item Depending on the $dst$ port, we generate either a packet drop action (line~\ref{Alg:LineGenDropRule}) or a packet forwarding action with the appropriate mapped port number of $dst.port$ (line~\ref{Alg:LineGenForwardRule1}-\ref{Alg:LineGenForwardRule2}).
\end{itemize}



\section{Evaluation}

\subsection{Experiment Scenario}


\subsection{Experiment Results}

\subsubsection{Preserve Forwarding Logic}


\subsubsection{Resource Reduction}


\section{Related Work}

\subsection{Network Forwarding Rules Abstraction}
This work is initially inspired by the "one big switch" idea proposed
in \cite{OneBigSwitchAbstraction}.
The entire SDN network are logically abstracted as one big OpenFlow switch such
that network developers just need to specify the end-to-end policies they want
to install into the network.
Underlying, the lower layers are responsible for address various implementation
issues such as hop-by-hop routing rules and rule placement mechanism.
In their work, the end-to-end policy is given by network applications.
Our work actually takes the reversed direction: we assume the forwarding rules on each
switch are available and we want to correctly infer the end-to-end policies.
Our three-step approach borrows several ideas like equivalence class, forwarding graphs
from VeriFlow\cite{Veriflow}, to slice the class of packets in the network.
There are, however, many significant differences between our approach and VeriFlow.
First VeriFlow uses EC and forwarding graphs so that verifying a new coming rule can be
both efficient and correct;
our goal is to generate rule on the big switch per packet class.
Secondly, we consider the entire connected SDN network when we run forwarding traversal,
especially the boundary switches that Veriflow neglects.
We also add up several algorithms in finding equivalence classes.

\subsection{SDN Emulation and Simulation}
There are a number of SDN emulation and simulation testbeds based on the OpenFlow
protocol.
Examples include Mininet\cite{Mininet}, EstiNet\cite{Estinet}, ns-3\cite{NS3},
S3FNet\cite{S3F}, fs-sdn\cite{FSSDN} and OpenNet\cite{OpenNet}.
The most popular SDN emulator Mininet applies container-based virtualization technique
and cgroup based resource isolation to provide lightweigth and high fidelity emualtion
platform.
Its functional fidelity is guaranteed by executing real SDN switch and controller software.
The traditional network simulator ns-3 supports models to simulate SDN network or emulate SDN
controller via direct code execution technique.
S3FNet is a hybrid OpenFlow-based SDN testing platform that integrates a parallel
network simulator with a OpenVZ-based network emulator.
fs-sdn is built on fs, a flow-level discrete event network simulator, and extends SDN
capability to fs.
We propose a systematic method to simplify the large scale and complicated SDN network target.
It serves as a preprocessing tool that generate a compact network model.
This resulting big-switch network can be used in both simulation and emulation environment
and its usage is not limited by any particular emulator or simulator.

\section{Conclusion and Future Work}
\label{Sec:conclusion}

We present a three-step model abstraction technique to transform an SDN-based network to
an ``one-big-switch" based network without losing the forwarding behavior as
defined by the OpenFlow rules in the network devices.
%We proposed a three-step solution to this abstraction problem. First we slice all the possible packets in the network into equivalence classes. Then we build forwarding graphs for each equivalence class. By traversing forwarding graphs, we can generate new rules to be installed on the big switch.
Experimental results demonstrate that the big-switch abstraction correctly models
the end-to-end forwarding logic of the original SDN network,
and the abstracted model significantly saves the experiment running time and system resources.
The ultimate goal of the one-big-switch abstraction is to enhance simulation and emulation scalability while preserving packet-level fidelity.
This paper mainly focuses on the end-to-end for- warding logic equivalence, and we will investigate end-to-end performance equivalence, such as latency and packet drop in the future.

\if 0
\hl{
While this paper mainly focuses on the end-to-end forwarding logic equivalence,
we also proceeded to investigate the performance equivalence,
including end-to-end latency and packet drop rate.
The basic idea is to collect the statistical information from the OpenFlow table entries
and analyze the random process of packet processing for each flow using queueing models.
Further details will be illustrated in future works.
}
\fi


\if 0 %removed temporarily for double-blinded review

%ACKNOWLEDGMENTS are optional
\section{Acknowledgments}
%Omitted for double-blind reviewing

This paper is partly sponsored by the Maryland Procurement Office under Contract No. H98230-14-C-0141, and the Air Force Office of Scientific Research (AFOSR) under grant FA9550-15-1-0190. Any opinions, findings and conclusions or recommendations expressed in this material are those of the author(s) and do not necessarily reflect the views of the Maryland Procurement Office and AFOSR.

\fi

%\end{document}  % This is where a 'short' article might terminate

%
% The following two commands are all you need in the
% initial runs of your .tex file to
% produce the bibliography for the citations in your paper.
\bibliographystyle{abbrv}
\bibliography{sigproc}  % sigproc.bib is the name of the Bibliography in this case
% You must have a proper ".bib" file
%  and remember to run:
% latex bibtex latex latex
% to resolve all references
%
% ACM needs 'a single self-contained file'!
%

%\nocite{*}
%\balancecolumns % GM June 2007
% That's all folks!
\end{document}
