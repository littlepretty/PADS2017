\section{Introduction}

% Bring in the issues that our approach can address
% Describe them as problems in current simulation/emulation systems

% State our idea
Inspired by works of~\cite{OneBigSwitchAbstraction, Veriflow},
we propose to abstract any subset of the SDN network switches as \textbf{``one big switch"}.
In the very extreme case, the entire set of the SDN switches are compressed to
one OpenFlow switch.
The obvious benefit is the much less required simulating or emulating resources,
for example, switch processes and number of rules existing in the network.
Besides, the abstracted big switch can be used in multiple plug-in-play scenarios.
For example, researchers can reproduce the simulation result with very
simple configurations (link connectivity setup, flow table configuration etc.)
after one complex run on the original network.
In the case of a too-large-to-simulate network, we can divide it into subsets of switches,
abstract each subset separately and then combine them together, resulting in a
size-reduced network.

The big switch must preserve \textbf{packet-level fidelity} so that
it can be used with confidence in simulation or emulation.
More concretely, we need to abstract the SDN network under two constraints:
\begin{itemize}
\item \textbf{Logical equivalence.} The forwarding behavior of any packet must be identical
        between both representations of the real network. Packets being
        forwarded through out the network (1) will also be forwarded through out
        some interface of the big switch, (2) any modifications made by
        intermediate switches must be made by the big switch in proper order.
        Packets being dropped at some point in the network path will be drop by the
        big switch.
\item \textbf{Performance equivalence.} A packet transmitted through a series of
        switches can be seen as processed by multiple queueing systems $\mathcal{Q}$.
        Similarly, the packet processing performance model in a big OpenFlow switch can
        also be modeled as a queueing system $Q$.
        The parameters of the later should be configured so that per packet performance
        measurements are as closed as possible to the ones in the former composite model.
        As an informal example, queueing delay for packet $p$, $Q(p, T)$, should be an
        function that approximates $\mathcal{Q}(p, T')$.
\end{itemize}

While our long term goal is to propose systematic approaches and algorithms to reduce
networked SDN switches to one switch that satisfies both logical equivalence and performance
equivalence, in this work, we mainly focus on the former constraint.
Our contributions include:
\begin{itemize}
\item We apply the reversed idea of ``One Big Switch"\cite{OneBigSwitchAbstraction}
        to SDN network simulation or emulation in order to
        improve their scalability and reusability.
\item Built on the idea of equivalence class and forwarding graph, we design
        a systematic approach to compress all the rules in a SDN network without
        loss of forwarding logic.
\item For several phases in our approach, we propose corresponding optimization
        algorithms to reduce the time complexity.
\end{itemize}


