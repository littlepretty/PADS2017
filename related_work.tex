\section{Related Work}
\label{Sec:relatedwork}

\subsection{SDN Forwarding Rules Abstraction}
\hl{
The idea of ``one big switch" orginate from~\mbox{\cite{OneBigSwitchAbstraction}} but the authors
proposed it for a different purpose.
}
In their work, the network abstraction is used to reduce conflicting rules generated by various
high-level SDN applications, which often simultaneously run on one or even multiple controllers.
Application developers only need to specify the end-to-end connectivity policies on the big switch.
\hl{
To accomplish this goal, the SDN infrastructure~\mbox{\cite{OneBigSwitchAbstraction}}
is given an network abstraction model and various algorithms are developed to calculate hop-by-hop routing,
schedule rule placements and install rules on each individual openflow switch.
Our work is motivated for the purpose of enhancing the scalability of network simulation and emulation.
What we want to abstract from the target network is in fact what is given to the
infrastructure developed by~\mbox{\cite{OneBigSwitchAbstraction}}.
}
We collect the forwarding rules on each device and use the rules to correctly infer the end-to-end forwarding policies.
Our model abstraction approach is based on statically analyzing snapshots of the network state.
There exists a line of research on network fault detection by analyzing software,
configuration and network-wide data-plane state~\cite{Al-Shaer2010,Al-Shaer2009,Anteater2011,xz+05}.
Those approaches typically operate offline on timescales of seconds to hours.
Real time network verification tools are developed to enforce correctness in connectivity \cite{NetPlumber2013,Veriflow}.
\hl{
NetPlumber~\mbox{\cite{NetPlumber2013}} uses a ``header space analysis" model to
describe forwarding behaviors as tranformation on arbitrary header bits.
It has very elegant geometric interpretation but, unlike Veriflow~\mbox{\cite{Veriflow}}, did not
provide data structures to aggregate or organize the classes of packets.
}
Our work leverages the idea of slicing the entire network into equivalence classes in \cite{Veriflow} to reduce the problem space, which enables fast model abstraction execution speed.
%We consider the entire connected SDN network when we run forwarding traversal, especially the boundary switches that Veriflow neglects. We also add up several algorithms in finding equivalence classes.


\subsection{SDN Emulation and Simulation}
There are a number of SDN emulation and simulation testbeds based on the OpenFlow
protocol.
Examples include Mininet~\cite{Mininet}, EstiNet~\cite{Estinet}, ns-3~\cite{NS3},
S3FNet \cite{S3F_website}, fs-sdn \cite{FSSDN} and OpenNet \cite{OpenNet}.
Mininet  \cite{Mininet} applies container-based virtualization technique and cgroup based resource isolation to provide a lightweight and high fidelity emulation
platform.
Its functional fidelity is guaranteed by executing real SDN switch/controller software.
ns-3 \cite{NS3} offers simulation models of SDN networks and emulation of SDN controllers via the direct code execution (DCE) technique.
S3FNet \cite{S3F_website} is a hybrid OpenFlow-based SDN testing platform that integrates a parallel network simulator with an OpenVZ-based network emulator.
fs-sdn \cite{FSSDN} extends fs, a flow-level discrete event network simulator, with the SDN capability.
We develop a model abstraction method in this paper to transform a large scale and complicated SDN network to a one-big-switch-based network.
We can use the resulting abstracted network model in all the aforementioned simulation and emulation environment for performance gain while still preserving the network forwarding logic.
